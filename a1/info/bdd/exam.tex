\documentclass[landscape]{article}

\usepackage[utf8]{inputenc}
\usepackage[T1]{fontenc}
\usepackage[francais]{babel}
\usepackage[left=1cm, right=1cm, bottom=1cm, top=1cm]{geometry}
\usepackage{fixltx2e}
\usepackage{fancyhdr}
\usepackage{graphicx}
\usepackage{amsmath}
\usepackage{amssymb}
\usepackage{mathrsfs}
\usepackage{stmaryrd}
\usepackage{bbm}
\usepackage[dvipsnames]{xcolor}
\usepackage[object=vectorian]{pgfornament}
\usetikzlibrary{shapes.geometric,calc}
\usepackage{algorithm2e}
\usepackage{multicol}
\newcommand{\sep}{\begin{center}
    \pgfornament[width=0.25\linewidth]{84}
\end{center}}
\newcommand{\edornament}{\begin{center}
    \pgfornament[width=0.25\linewidth]{49}
\end{center}}
\newcommand{\seg}[2]{\llbracket #1, #2 \rrbracket}

\MakeRobust{\overrightarrow}

\newcommand{\vc}[1]{\overrightarrow{#1}}
\newcommand{\A}{\mathscr{A}}
\newcommand{\F}{\mathscr{F}}
\newcommand{\B}{\mathscr{B}}
\newcommand{\C}{\mathscr{C}}
\newcommand{\G}{\mathscr{G}}
\renewcommand{\L}{\mathscr{L}}
\newcommand{\V}{\mathscr{V}}
\newcommand{\T}{\mathscr{T}}
\newcommand{\Q}{\mathscr{Q}}
\renewcommand{\H}{\mathscr{H}}
\newcommand{\Er}{\mathscr{E}}
\newcommand{\parts}{\mathscr{P}}
\newcommand{\ntoi}{{n\rightarrow\infty}}
\newcommand{\N}{\mathbb{N}}
\newcommand{\Z}{\mathbb{Z}}
\newcommand{\R}{\mathbb{R}}
\newcommand{\E}{\mathbb{E}}
\newcommand{\1}{\mathbbm{1}}
\renewcommand{\phi}{\varphi}

\newtheorem{theo}{Théorème}
\newtheorem{lem}{Lemme}
\newtheorem{prop}{Proposition}
\newtheorem{defi}{Définition}
\newtheorem{fact}{Faits}

\begin{document}
\begin{multicols}{2}

    \section{Introduction}

    \begin{defi}[SGBD] Fonctionnalités : indépendance physique/logique, accès aisé
        aux données, optimisation de requêtes, intégrité logique/physique,
        partage de données, normalisation.
    \end{defi}

    \begin{defi}[ACID] Atomicité, Cohérence, Isolation, Durabilité.
    \end{defi}

    \begin{fact}[Faiblesses SGBD classiques] Très grands volumes (> Po), débits
        extrêmes (> kRequêtes/s), mal adapté à certains modèles de données, surcoût
        de ACID.
    \end{fact}

    \begin{defi}[Schéma de relation d'arité $n$]0
        Soient $\L$ les étiquettes, $\V$ les valeurs et $\T\subset\P(\V)$ les types
        des ensembles infinis dénombrables. Un schéma de relation est un $n$-uplet
        $(A_i)_{1\leq i \leq n}\in(\L\times \T)^n$ tel que $A_i\neq A_j$ pour tout
        $1\leq i,j\leq n$.
    \end{defi}

    \begin{defi}[Schéma relationnel] Ensemble de schéma de relations étiquetés.
    \end{defi}

    \begin{defi}[Contraintes d'intégrité] Pour une instance :\begin{itemize}
        \item Clef : attributs d'une relation permettant d'identifier les
            éléments de l'instance.
        \item Clef étrangère : attributs d'une relation permettant d'identifier
            (existence + unicité) les éléments d'une autre relation.
        \item Contrainte Check : condition arbitraire sur les valeurs des attributs
            d'une relation.
    \end{itemize}\end{defi}

    \section{Aspects logiques}

    \begin{defi}[Algèbre relationelle : opérateurs]
        \begin{tabular}{ccll}
            \hline
            Op. & Arité & Description & Condition \\
            \hline
            $R$ & 0 & Nom de relation & $R \in \L$ \\
            $\rho_{A\rightarrow B}$ & 1 & Renommage & $A,B\in\L$ \\
            $\Pi_{A_1,\ldots A_n}$ & 1 & Projection & $A_i\in\L$ \\
            $\sigma_\phi$ & 1 & Sélection & $\phi$ formule \\
            $\times$ & 2 & Produit cartésien & \\
            $\cup$ & 2 & Union & \\
            $\backslash$ & 2 & Différence & \\
            $\bowtie_\phi$ & 2 & Jointure & $\phi$ formule \\
            \hline
        \end{tabular}
    \end{defi}

    \begin{defi}[Requêtes] \begin{itemize}
        \item Requêtes conjonctives : $x_a, x_b \leftarrow R_1(x, x_a), R_2(x', x_b, x''), R_3(x_b, x'')$
        \item Calcul conjonctif : $\{x_a, x_b | \exists x \exists x' \exists x''(R_1(x, x_a)\wedge R_2(x', x_b, x'')\wedge R_3(x_b, x'')) \}$
        \item Algèbre PSJR : projection, sélection, jointure, renommage.
    \end{itemize}\end{defi}

    \begin{theo}[Théorème d'équivalence PSJR]
        Les requêtes conjonctives, le calcul conjonctif et l'algèbre PSJR permettent
        d'exprimer les mêmes requêtes.
    \end{theo}

    \begin{theo}[Théorème d'équivalence PSJRU]
        L'équivalence reste vraie si on ajoute l'union à l'algèbre, le $\vee$ au
        calcul et la possibilité de mettre plusieurs requêtes.
    \end{theo}

    \begin{defi}[Sémantique domaine active]
        On défini le domaine actif d'une instance comme l'ensemble des constantes.
        Cette sémantique de requêtes ne renvoit que des tuples contenant des éléments
        du domaine actif de l'instance.
    \end{defi}

    \begin{theo}[Théorème d'équivalence]
        Algèbre, calcul et datalog (avec $\backslash$, $\forall$, et $\neg$) sont
        équivalents.
    \end{theo}

    \section{Requêtes récursives}

    % TODO ajouter récursivité et points fixes ?

    \section{Complexité des langages de requête}

    \begin{defi}[Capture] Un langage de requête $\Q$ capture un classe de complexité
        $\C$ si :\begin{itemize}
            \item pour tout $Q\in\Q$, évaluer $Q$ est dans $\C$ en complexité
                en les données.
            \item pour tout $P\in\C$, il existe $Q\in\Q$ dont l'évaluation
                résoud $P$.
        \end{itemize}
    \end{defi}

    \begin{theo}[Classes de complexité]
        \begin{tabular}{r|cc}
            \hline Langage & donnés & combinée \\
            \hline
            $CQ$ & $PTIME$ & $NP$-complet \\
            $FO$ & $AC0\subset PTIME$ & $PSPACE$-complet \\
            $FO+\mu^+$ & $PTIME$-complet & \\
            $FO+\mu$ & $PSPACE$-complet & \\
            \hline
        \end{tabular}
    \end{theo}

    \begin{defi}[$\alpha$-acyclicité]
        Soit $R$ une requête conjonctive. $R$ peut être vue comme un hypergraphe $\H$
        dont les sommets sont les variables et les hyperarêtes les atomes. $R$ est
        $\alpha$-acyclique si $H$ admet un arbre de jointure, ie un arbre vérifiant :
        \begin{itemize}
            \item chaque hyperarête de $\H$ apparait comme étiquette d'un noeud de l'arbre.
            \item pour chaque sommet $x$ de $\H$, l'ensemble des noeuds de l'arbre étiquetés
                par une arêtes incluant $x$ est un sous-arbre connexe.
        \end{itemize}
    \end{defi}

    \begin{theo}[Théorème de Tarjan et Yannakabis]
        L'arbre de jointure d'une requête peut être obtenu en temps linéaire
        s'il existe.
    \end{theo}

    \begin{theo}[Algorithme de Yannakabis]
        % TODO détail de l'algorithme
        Une requête $\alpha$-acyclique s'évalue en temps combiné polynomial.
    \end{theo}

    \begin{theo} $FO+\mu^+$ capture $PTIME$ sur les données avec un ordre total sur
        les éléments du domaine.
        
        $FO+\mu$ capture $PSPACE$ sur les données avec un ordre total sur les éléments
        du domaine.
    \end{theo}

    \section{Contraintes et poursuite}

    \begin{defi}[Dépendance fonctionnelle (FD)]
        Une instance $r$ vérifie $X\rightarrow Y$, noté $x\models X\rightarrow Y$,
        si $\forall t, t'\in r, t[X] = t[X']\implies t[y] = t'[y]$.
    \end{defi}

    \begin{defi}[Dépendance multivaluée (MVD)]
        $r\models X\twoheadrightarrow Y$ si, pour $Z$ les attributs de $R$ hors
        $X\cup Y$, $\forall t, t'\in R, t[X] = t[X'] \implies \exists t''\in r:
        t''[Y] = t[y] \wedge t''[Z] = t'[Z]$
    \end{defi}

    \begin{defi}[Algorithme de poursuite]
        Décide si une dépendance est impliquée par un ensemble de dépendances, en
        partant d'une instance pour une relation avec deux éléments, puis en ajoutant
        des éléments en utilisant les dépendances initiale jusqu'à obtenir une instance
        vérifiant ces dépendances. Si la dépendance finale est vérifiée, alors
        elle est impliquée, sinon l'instance créée est un contre-exemple.
    \end{defi}

    \begin{theo}[Maier]
        La poursuite par des FD et MVD termine toujours après un nombre fini
        d'opérations, et contruit un contre-exemple si et seulement s'il en
        existe un.
    \end{theo}

    \begin{defi}[Dépendances génératrices d'égalités (EGD)]
        $\forall x_1,\ldots x_n, \phi(x_1,\ldots x_n)\implies \psi(x_1,\ldots x_n)$
        avec $\phi$ une CQ et $\psi$ une conjonction d'égalités. Les EGD généralisent
        les FD.
    \end{defi}

    \begin{defi}[Dépendances génératrices de tuples (TGD)]
        $\forall x_1,\ldots x_n, \phi(x_1,\ldots x_n)\implies\exists y_1,\ldots y_m, \psi(x_1,\ldots x_n, y_1, \ldots y_m)$
        avec $\phi$ et $\psi$ des CQ. Les TGD généralisent les MVD et les ID
        (clefs étrangères).
    \end{defi}

    % TODO poursuite générale

    \section{Théorie de la normalisation}
    
    % TODO 3NF et BCNF definitions

    \section{Analyse statique de requêtes}

    \begin{defi}[Inclusion]
        Une requête $q$ est incluse dans une requête $q'$ (noté $q\sqsubseteq q'$)
        si pour toute base de données $D$, $q(D)\subseteq q'(D)$.
    \end{defi}

    \begin{prop}[Équivalence et inclusion]
        $q\equiv q' \iff q\sqsubseteq q'\wedge q'\sqsubseteq q$
    \end{prop}

    \begin{defi}[Homomorphisme]
        Un homomorphisme $\phi$ d'une CQ $q$ dans une CQ $q'$ est une fonction
        des variables de $q$ dans celles de $q'$ telle que pour tout atome
        $R(z_i)$ de $q$, il existe un atome $R(z'_{i'})$ dans $q'$ avec
        $\phi(z_i) = z'_{i'}$.

        Un isomorphisme est un homomorphisme bijectif dont la bijection réciproque
        est un homomorphisme.
    \end{defi}

    \begin{defi}[Instance associée à une requête] Soit $q(x) \leftarrow \exists y:
        R_1(z_1)\wedge\ldots\wedge R_n(z_n)$ une CQ. On peut construire $I_q$
        sont instance associée de domaine actif $\{a_z | z\in x\cup y\}$
        formée des $n$ tuples $R(a_{z_{i1}}, \ldots a_{z_{ik}})$ pour
        $R(z_{i1}, \ldots z_{ik})$ atome de $q$.
    \end{defi}

    \begin{prop} Il existe une homomorphisme de $q(x)$ dans $q'(x')$ ssi
        $(a_{x'_1}, \ldots a_{x'_j})\in q(I_{q'})$.
    \end{prop}

    \begin{theo}[Chandra et Merlin] $q\sqsubseteq q'$ ssi il existe un homomorphisme
        de $q'$ dans $q$.
    \end{theo}

    \begin{prop}[Chandra et Merlin] Soit $q$ une CQ. Il existe une requête $q'$
        équivalente minimale à $q$ obtenue en enlevant des atomes à $q$.

        Soient $q,q'$ deux CQ équivalentes minimales. Il existe un isomorphisme
        de $q$ dans $q'$.
    \end{prop}

    \begin{prop} Les problèmes suivants sont $NP$-complets (en les requêtes)~: $q\sqsubseteq q'$,
        $q \equiv q'$ et la non minimalité de $q$.
    \end{prop}

    \begin{theo}[Trakhtenbrot]
        La satisfiabilité du calcul relationnel (dans le cadre fini) est indécidable.
    \end{theo}

    \begin{theo} L'inclusion et l'équivalence de requêtes du calcul relationnel
        sont indécidables et co-récursivement énumérables.
    \end{theo}

\end{multicols}
\end{document}

